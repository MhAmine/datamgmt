\nonstopmode{}
\documentclass[letterpaper]{book}
\usepackage[times,inconsolata,hyper]{Rd}
\usepackage{makeidx}
\usepackage[utf8]{inputenc} % @SET ENCODING@
% \usepackage{graphicx} % @USE GRAPHICX@
\makeindex{}
\begin{document}
\chapter*{}
\begin{center}
{\textbf{\huge Package `datamgmt'}}
\par\bigskip{\large \today}
\end{center}
\begin{description}
\raggedright{}
\inputencoding{utf8}
\item[Type]\AsIs{Package}
\item[Title]\AsIs{Data Management Utilities for Curating, Documenting, and
Publishing Data}
\item[Version]\AsIs{0.1.0}
\item[Author]\AsIs{Who wrote it}
\item[Maintainer]\AsIs{The package maintainer }\email{yourself@somewhere.net}\AsIs{}
\item[Description]\AsIs{More about what it does (maybe more than one line)
Use four spaces when indenting paragraphs within the Description.}
\item[License]\AsIs{Apache License (== 2.0)}
\item[Encoding]\AsIs{UTF-8}
\item[LazyData]\AsIs{true}
\item[Suggests]\AsIs{testthat}
\item[RoxygenNote]\AsIs{6.0.1}
\end{description}
\Rdcontents{\R{} topics documented:}
\inputencoding{utf8}
\HeaderA{add\_creator\_id}{add\_creator\_id}{add.Rul.creator.Rul.id}
\keyword{creator}{add\_creator\_id}
\keyword{eml}{add\_creator\_id}
\keyword{id}{add\_creator\_id}
\keyword{orcid}{add\_creator\_id}
%
\begin{Description}\relax
This function allows you to add an ORCID or reference ID to a creator in EML.
\end{Description}
%
\begin{Usage}
\begin{verbatim}
add_creator_id(eml, orcid = NULL, id = NULL, surname = NULL)
\end{verbatim}
\end{Usage}
%
\begin{Arguments}
\begin{ldescription}
\item[\code{eml}] EML script to modify

\item[\code{orcid}] ORCID in the format 'https://orcid.org/WWWW-XXXX-YYYY-ZZZZ'

\item[\code{id}] reference ID. Character string to set reference ID for creators with additional roles (i.e. metadataProvider, etc.)

\item[\code{surname}] creator surname (last name), defaults to first creator if not specified. Not case-sensitive.
\end{ldescription}
\end{Arguments}
%
\begin{Details}\relax
The function invisibly returns the full EML, which
can be saved to a variable. It also prints the changed creator
entry so that it's easy to check that the appropriate change was
made. All prameters other than the EML are optional, but since
the point of the function is to modify either the orcid, ref id,
or both, you need to specify at least one. Requires the
crayon package.
\end{Details}
%
\begin{Examples}
\begin{ExampleCode}
library(dataone)
library(arcticdatautils)
library(EML)

cnTest <- dataone::CNode('STAGING')
mnTest <- dataone::getMNode(cnTest,'urn:node:mnTestARCTIC')
eml_pid <- arcticdatautils::create_dummy_metadata(mnTest)
eml1 <- EML::read_eml(rawToChar(getObject(mnTest, eml_pid)))
add_creator_id(eml1, orcid = "https://orcid.org/WWWW-XXXX-YYYY-ZZZZ")
\end{ExampleCode}
\end{Examples}
\inputencoding{utf8}
\HeaderA{build\_custom\_units}{Build shiny custom units table}{build.Rul.custom.Rul.units}
%
\begin{Description}\relax
Build shiny custom units table
\end{Description}
%
\begin{Usage}
\begin{verbatim}
build_custom_units(inputdf, standardUnits, inputdf2, unq_unitType, unq_parentSI)
\end{verbatim}
\end{Usage}
%
\begin{Arguments}
\begin{ldescription}
\item[\code{inputdf}] attributes table

\item[\code{standardUnits}] standardUnits table

\item[\code{inputdf2}] current units table

\item[\code{unq\_unitType}] unique standardUnits unitTypes

\item[\code{unq\_parentSI}] unique standardUnits parentSI
\end{ldescription}
\end{Arguments}
\inputencoding{utf8}
\HeaderA{build\_factors}{Build shiny factors table}{build.Rul.factors}
%
\begin{Description}\relax
Build shiny factors table
\end{Description}
%
\begin{Usage}
\begin{verbatim}
build_factors(inputdf, inputdf2, data)
\end{verbatim}
\end{Usage}
%
\begin{Arguments}
\begin{ldescription}
\item[\code{inputdf}] attributes table

\item[\code{inputdf2}] current factors table

\item[\code{data}] initial data inputed by user
\end{ldescription}
\end{Arguments}
\inputencoding{utf8}
\HeaderA{clone\_one\_package}{Clone a Data Package without its child packages.}{clone.Rul.one.Rul.package}
%
\begin{Description}\relax
The wrapper function 'clone\_package' should be used instead. This function
copies a data package from one DataOne member node to another, excluding any
child data packages.
\end{Description}
%
\begin{Usage}
\begin{verbatim}
clone_one_package(mn_pull, mn_push, resource_map_pid)
\end{verbatim}
\end{Usage}
%
\begin{Arguments}
\begin{ldescription}
\item[\code{mn\_pull}] (MNode) The Member Node to download from.

\item[\code{mn\_push}] (MNode) The Member Node to upload to.

\item[\code{resource\_map\_pid}] (chraracter) The identifier of the Resource Map for the package to download.
\end{ldescription}
\end{Arguments}
%
\begin{Value}
(list) List of all the identifiers in the new Data Package.
TODO switch remaining for loops to applys
TODO add more messages
TODO better way to set physical in Update Metadata section?
TODO pull/push terminology could potentially be confusing. perhaps consider download/upload, from/to, source/new could be better?  I do like that they match well (both 4-letter p-words)
TODO since messages print in red (scary!), you might want to consider the crayon workaround you found. maybe it's worth having a discussion on our package 'style'?
\end{Value}
%
\begin{Author}\relax
Dominic Mullen, \email{dmullen17@gmail.com}
\end{Author}
\inputencoding{utf8}
\HeaderA{clone\_package}{Clone a Data Package}{clone.Rul.package}
%
\begin{Description}\relax
This function copies a Data Package from one DataOne member node to another.
It can also be used to copy an older version of a Data Package to the same
member node in order to restore it, provided that the old Package is then
obsoleted by the copied version.
\end{Description}
%
\begin{Usage}
\begin{verbatim}
clone_package(mn_pull, mn_push, resource_map_pid)
\end{verbatim}
\end{Usage}
%
\begin{Arguments}
\begin{ldescription}
\item[\code{mn\_pull}] (MNode) The Member Node to download from.

\item[\code{mn\_push}] (MNode) The Member Node to upload to.

\item[\code{resource\_map\_pid}] (chraracter) The identifier of the Resource Map for the package to download.
\end{ldescription}
\end{Arguments}
%
\begin{Author}\relax
Dominic Mullen, \email{dmullen17@gmail.com}
\end{Author}
%
\begin{Examples}
\begin{ExampleCode}
## Not run: 
cn_pull <- CNode("PROD")
mn_pull <- getMNode(cn_pull, "urn:node:ARCTIC")
cn_push <- CNode('STAGING')
mn_push <- getMNode(cn_push,'urn:node:mnTestARCTIC')
clone_package(mn_pull, mn_push, "resource_map_doi:10.18739/A2RZ6X")

## End(Not run)

\end{ExampleCode}
\end{Examples}
\inputencoding{utf8}
\HeaderA{compare\_eml\_to\_package\_pids}{Return data object identifiers from 'dataTable' and 'otherEntity' objects in an EML file}{compare.Rul.eml.Rul.to.Rul.package.Rul.pids}
%
\begin{Description}\relax
This function is a helper function for 'clone\_package'.  It checks that all
of the data objects pids present in the metadata match those that the
resource map points to.  It also returns the data pids in the order in which
they appear in the EML: dataTable pids first and then otherEntity pids.  We
cannot update the new .xml file without this information.
\end{Description}
%
\begin{Usage}
\begin{verbatim}
compare_eml_to_package_pids(eml, package_data_pids)
\end{verbatim}
\end{Usage}
%
\begin{Arguments}
\begin{ldescription}
\item[\code{eml}] (EML) EML object

\item[\code{package\_data\_pids}] (character) All of the Data identifiers in a Data
Package.  These can be selected with get\_package(mn, resource\_map)\$data
\end{ldescription}
\end{Arguments}
%
\begin{Value}
(list) List of all the identifiers in the EML.  Sorted into dataTable
and otherEntity identifiers
TODO better name for this function?
\end{Value}
%
\begin{Author}\relax
Dominic Mullen, \email{dmullen17@gmail.com}
\end{Author}
\inputencoding{utf8}
\HeaderA{create\_attributes\_table}{Allows editing of an attribute table and custom units table in a shiny environment}{create.Rul.attributes.Rul.table}
%
\begin{Description}\relax
Allows editing of an attribute table and custom units table in a shiny environment
\end{Description}
%
\begin{Usage}
\begin{verbatim}
create_attributes_table(data = NULL, attributes_table = NULL)
\end{verbatim}
\end{Usage}
%
\begin{Arguments}
\begin{ldescription}
\item[\code{data}] The data.frame of data that needs an attribute table

\item[\code{attributes\_table}] A existing attributes table for \code{data} that needs to be updated. If specified, all non empty fields will be used (i.e. if numberType is specified in \code{attributes\_table}, then this function will use those values instead of automatically generating values from \code{data}).
\end{ldescription}
\end{Arguments}
%
\begin{Examples}
\begin{ExampleCode}
create_attributes_table(NULL, NULL)

data <- read.csv("Test.csv")
create_attributes_table(data, NULL)

attributes_table <- EML::get_attributes(eml@dataset@dataTable[[i]]@attributeList)$attributes
create_attributes_table(NULL, attributes_table)

create_attributes_table(data, attributes_table)
\end{ExampleCode}
\end{Examples}
\inputencoding{utf8}
\HeaderA{data\_objects\_exist}{Check if data objects exist in a list of Data Packages.}{data.Rul.objects.Rul.exist}
%
\begin{Description}\relax
This function is primarily intended to assist Mark Schildhauer in preparation
for the Carbon synthesis working group.
\end{Description}
%
\begin{Usage}
\begin{verbatim}
data_objects_exist(mn, pids, write_to_csv = FALSE, folder_path = NULL,
  file_name = NULL)
\end{verbatim}
\end{Usage}
%
\begin{Arguments}
\begin{ldescription}
\item[\code{mn}] (MNode/CNode) The Node to query for Object sizes

\item[\code{pids}] (character) The identifier of the Data Packages' Metadata

\item[\code{write\_to\_csv}] (logical) Optional. Write the query results to a csv

\item[\code{folder\_path}] (character) Optional. Folder to write results to

\item[\code{file\_name}] (character) Optional. Name of results file in csv format
\end{ldescription}
\end{Arguments}
%
\begin{Value}
(data.frame) Data frame containing query results.
\end{Value}
%
\begin{Author}\relax
Dominic Mullen, \email{dmullen17@gmail.com}
\end{Author}
%
\begin{Examples}
\begin{ExampleCode}
## Not run: 
cn <- CNode("PROD")
mn <- getMNode(cn, "urn:node:ARCTIC")
data_objects_exist(mn,
c("doi:10.5065/D60P0X4S", "urn:uuid:3ea5629f-a10e-47eb-b5ce-de10f8ef325b"))

## End(Not run)

\end{ExampleCode}
\end{Examples}
\inputencoding{utf8}
\HeaderA{excel\_to\_csv}{Convert excel workbook to multiple csv files}{excel.Rul.to.Rul.csv}
%
\begin{Description}\relax
Converts an excel workbook into multiple csv files (one per tab).  Names the
files in the following format: sheetName\_excelName.csv.
\end{Description}
%
\begin{Usage}
\begin{verbatim}
excel_to_csv(path, directory = NULL, ...)
\end{verbatim}
\end{Usage}
%
\begin{Arguments}
\begin{ldescription}
\item[\code{path}] (character) File location of the excel workbook.

\item[\code{directory}] (character) Optional.  Directory to download csv files to.
Defaults to the base directory that \code{path} is located in.
\end{ldescription}
\end{Arguments}
%
\begin{Value}
(invisible())
\end{Value}
%
\begin{Author}\relax
Dominic Mullen \email{dmullen17@gmail.com}
\end{Author}
\inputencoding{utf8}
\HeaderA{factor\_att\_table}{Creates factors for attributes table}{factor.Rul.att.Rul.table}
%
\begin{Description}\relax
Creates factors for attributes table
\end{Description}
%
\begin{Usage}
\begin{verbatim}
factor_att_table(inputdf)
\end{verbatim}
\end{Usage}
%
\begin{Arguments}
\begin{ldescription}
\item[\code{inputdf}] attributes table
\end{ldescription}
\end{Arguments}
\inputencoding{utf8}
\HeaderA{get\_eml\_pids}{Return data identifiers from within an EML}{get.Rul.eml.Rul.pids}
%
\begin{Description}\relax
This function returns data object identifers present within an EML (electronic
metadata language) document.
\end{Description}
%
\begin{Usage}
\begin{verbatim}
get_eml_pids(eml)
\end{verbatim}
\end{Usage}
%
\begin{Arguments}
\begin{ldescription}
\item[\code{eml}] () an EML object
\end{ldescription}
\end{Arguments}
%
\begin{Value}
Returns a list of data object pids in the eml
\end{Value}
\inputencoding{utf8}
\HeaderA{get\_numberType}{Returns the numberType (either "real", "integer", "whole", or "natural") of input values}{get.Rul.numberType}
%
\begin{Description}\relax
Returns the numberType (either "real", "integer", "whole", or "natural") of input values
\end{Description}
%
\begin{Usage}
\begin{verbatim}
get_numberType(values)
\end{verbatim}
\end{Usage}
%
\begin{Arguments}
\begin{ldescription}
\item[\code{values}] A vector of values. If vector is non-numeric will return NA
\end{ldescription}
\end{Arguments}
%
\begin{Value}
The numberType of \code{values} (either "real", "integer", "whole", or "natural").
\end{Value}
%
\begin{Examples}
\begin{ExampleCode}
# To get numberType for each column in a data.frame \code{df}:
unlist(lapply(df, function(x) get_numberType(x)))
\end{ExampleCode}
\end{Examples}
\inputencoding{utf8}
\HeaderA{get\_object\_metadata}{Return the object metadata of each data object in a Data Package}{get.Rul.object.Rul.metadata}
%
\begin{Description}\relax
This function returns the formatId, fileName and identifier of each data
object in a Data Package.  This is a helper function for the function
'clone\_package' - which copies a dataOne Data Package from one member node to
another.
\end{Description}
%
\begin{Usage}
\begin{verbatim}
get_object_metadata(mn, resource_map_pid, formatType = "DATA")
\end{verbatim}
\end{Usage}
%
\begin{Arguments}
\begin{ldescription}
\item[\code{mn}] (MNode/CNode) The Node to query for Object sizes

\item[\code{resource\_map\_pid}] (character) The identifier of the Data Package's Resource Map

\item[\code{formatType}] (character) Optional. Filter to just Objects of the given
formatType. One of METADATA, RESOURCE, or DATA or * for all types
\end{ldescription}
\end{Arguments}
%
\begin{Value}
(character) The formatId, fileName, and identifier of each data object in a package.
\end{Value}
%
\begin{Author}\relax
Dominic Mullen, \email{dmullen17@gmail.com}
\end{Author}
\inputencoding{utf8}
\HeaderA{hello}{Hello, World!}{hello}
%
\begin{Description}\relax
Prints 'Hello, world!'.
\end{Description}
%
\begin{Usage}
\begin{verbatim}
hello()
\end{verbatim}
\end{Usage}
%
\begin{Examples}
\begin{ExampleCode}
hello()
\end{ExampleCode}
\end{Examples}
\inputencoding{utf8}
\HeaderA{output\_text\_func}{Outputs data.frame to text for shiny app}{output.Rul.text.Rul.func}
%
\begin{Description}\relax
Outputs data.frame to text for shiny app
\end{Description}
%
\begin{Usage}
\begin{verbatim}
output_text_func(df)
\end{verbatim}
\end{Usage}
%
\begin{Arguments}
\begin{ldescription}
\item[\code{df}] data.frame
\end{ldescription}
\end{Arguments}
\inputencoding{utf8}
\HeaderA{shiny\_attributes\_table}{Build shiny UI for editing attributes table within function create\_attributes\_table()}{shiny.Rul.attributes.Rul.table}
%
\begin{Description}\relax
Build shiny UI for editing attributes table within function create\_attributes\_table()
\end{Description}
%
\begin{Usage}
\begin{verbatim}
shiny_attributes_table(att_table, data)
\end{verbatim}
\end{Usage}
%
\begin{Arguments}
\begin{ldescription}
\item[\code{att\_table}] an attribute table built from create\_attributes\_table()
\end{ldescription}
\end{Arguments}
\printindex{}
\end{document}
